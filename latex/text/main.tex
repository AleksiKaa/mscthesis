%%%%%%%%%%%%%%%%%%%%%%%%%%%%%%%%%%%%%%%%%%%%%%%%%%%%%%%%%%%%%%%%%%%%%%%%%%%%%%%%
%%%%%%%%%%%%%%%%%%%%%%%%%%%%%%%%%%%%%%%%%%%%%%%%%%%%%%%%%%%%%%%%%%%%%%%%%%%%%%%%
%%                                                                            %%
%% thesistemplate_short.tex version 4.10 (2025/06/30)                         %%
%% The LaTeX template file to be used with the aaltothesis.sty (version 4.10) %%
%% style file.                                                                %%
%% This package requires pdfx.sty v. 1.5.84 (2017/05/18) or newer.            %%
%%                                                                            %%
%% This is licensed under the terms of the MIT license below.                 %%
%%                                                                            %%
%% Written by Luis R.J. Costa.                                                %%
%% Currently developed at Teacher services, Learning Services of Aalto        %%
%% University by Luis R.J. Costa since May 2019.                              %%
%%                                                                            %%
%% Copyright 2017-2025 aaltothesis.cls by Luis R.J. Costa,                    %%
%% luis.costa@aalto.fi.                                                       %%
%% Copyright 2017-2018 Swedish translations in aaltothesis.cls by Elisabeth   %%
%% Nyberg and Henrik Wallén henrik.wallen@aalto.fi.                           %%
%% Copyright 2017-2018 Finnish documentation in the template opinnatepohja.tex%%
%% by Perttu Puska, perttu.puska@aalto.fi, and Luis R.J. Costa.               %%
%% Finnish documentation in the template opinnatepohja.tex translated from    %%
%% the English template documentation.                                        %%
%% Copyright 2025 English template thesistemplate.tex by Luis R.J. Costa,     %%
%% Maurice Forget, Henrik Wallén.                                             %%
%% Copyright 2018-2025 Swedish template kandidatarbetsbotten.tex by Henrik    %%
%% Wallen.                                                                    %%
%%                                                                            %%
%% Permission is hereby granted, free of charge, to any person obtaining a    %%
%% copy of this software and associated documentation files (the "Software"), %%
%% to deal in the Software without restriction, including without limitation  %%
%% the rights to use, copy, modify, merge, publish, distribute, sublicense,   %%
%% and/or sell copies of the Software, and to permit persons to whom the      %%
%% Software is furnished to do so, subject to the following conditions:       %%
%% The above copyright notice and this permission notice shall be included in %%
%% all copies or substantial portions of the Software.                        %%
%% THE SOFTWARE IS PROVIDED "AS IS", WITHOUT WARRANTY OF ANY KIND, EXPRESS OR %%
%% IMPLIED, INCLUDING BUT NOT LIMITED TO THE WARRANTIES OF MERCHANTABILITY,   %%
%% FITNESS FOR A PARTICULAR PURPOSE AND NONINFRINGEMENT. IN NO EVENT SHALL    %%
%% THE AUTHORS OR COPYRIGHT HOLDERS BE LIABLE FOR ANY CLAIM, DAMAGES OR OTHER %%
%% LIABILITY, WHETHER IN AN ACTION OF CONTRACT, TORT OR OTHERWISE, ARISING    %%
%% FROM, OUT OF OR IN CONNECTION WITH THE SOFTWARE OR THE USE OR OTHER        %%
%% DEALINGS IN THE SOFTWARE.                                                  %%
%%                                                                            %%
%%                                                                            %%
%%%%%%%%%%%%%%%%%%%%%%%%%%%%%%%%%%%%%%%%%%%%%%%%%%%%%%%%%%%%%%%%%%%%%%%%%%%%%%%%
%%                                                                            %%
%% A concise template in English. For more detailed instructions in the use   %%
%% of this template and LaTeX-specific example see the English and Finnish    %%
%% templates thesistemplate.tex and opinnaytepohja.tex.                       %%
%%                                                                            %%
%%%%%%%%%%%%%%%%%%%%%%%%%%%%%%%%%%%%%%%%%%%%%%%%%%%%%%%%%%%%%%%%%%%%%%%%%%%%%%%%
%%                                                                            %%
%%                                                                            %%
%% An example for writting your thesis using LaTeX                            %%
%% Original version and development work by Luis Costa, changes to the text   %% 
%% in the Finnish template by Perttu Puska.                                   %%
%% Support for Swedish added 15092014                                         %%
%% PDF/A-b support added on 15092017                                          %%
%% PDF/A-2 support added on 24042018                                          %%
%% Layout design and typesettin changed 15072021                              %%
%%                                                                            %%
%% This example consists of the files                                         %%
%%       thesistemplate.tex (version 4.10) (for text in English)              %%
%%       opinnaytepohja.tex (version 4.10) (for text in Finnish)              %%
%%       kandidatarbetsbotten.tex (version 1.20) (for text in Swedish)        %%
%%       thesistemplate_short.tex (version 4.10) (abridged for text in        %%
%%                                                English)                    %%
%%       aaltothesis.cls                                                      %%
%%       linediagram.pdf (graphics file)                                      %%
%%       curves.pdf      (graphics file)                                      %%
%%       ledspole.jpg    (graphics file)                                      %%
%%                                                                            %%
%%                                                                            %%
%% Typeset in Linux with                                                      %%
%% pdflatex: (recommended method)                                             %%
%%             $ pdflatex thesistemplate                                      %%
%%             $ pdflatex thesistemplate                                      %%
%%                                                                            %%
%%   The result is the file thesistemplate.pdf that is PDF/A compliant, if    %%
%%   you have chosen the proper \documenclass options (see comments below)    %%
%%   and your included graphics files have no problems.                       %%
%%                                                                            %%
%%                                                                            %%
%% Explanatory comments in this example begin with the characters %%, and     %%
%% changes that the user can make with the character %                        %%
%%                                                                            %%
%%%%%%%%%%%%%%%%%%%%%%%%%%%%%%%%%%%%%%%%%%%%%%%%%%%%%%%%%%%%%%%%%%%%%%%%%%%%%%%%
%%%%%%%%%%%%%%%%%%%%%%%%%%%%%%%%%%%%%%%%%%%%%%%%%%%%%%%%%%%%%%%%%%%%%%%%%%%%%%%%
%%
%%
%% USE one of the following three \documentclass set-ups:
%% * the first when using pdflatex to directly typeset your document in the
%%   chosen pdf/a format for online publishing (centred page layout),
%% * the second for one-sided printing your thesis with the layout (wide left 
%%   margin), or
%% * the third for two-sided printing.
%%
\documentclass[english, 12pt, a4paper, sci, utf8, a-2b, online]{../definitions/aaltothesis}
%\documentclass[english, 12pt, a4paper, elec, utf8, a-2b, print]{aaltothesis}
%\documentclass[english, 12pt, a4paper, elec, utf8, a-2b, print, twoside]{aaltothesis}

%% Use the following options in the \documentclass macro above:
%% your school: arts, biz, chem, elec, eng, sci
%% the character encoding scheme used by your editor: utf8, latin1
%% thesis language: english, finnish, swedish
%% make an archiveable PDF/A-1b or PDF/A-2b compliant file: a-1b, a-2b
%%                    (with pdflatex, a normal pdf containing metadata is
%%                     produced without the a-*b option)
%%                    NOTE: when using option a-1b, do not use doclicense
%%                    features to typeset the copyright text. The transparency
%%                    in the image is incompatible with the PDF/A-1 standard.
%% typset for online document or print on paper: online, print
%%        online: typeset in symmetric layout and blue hypertext for online
%%                publishing
%%        print: typeset in a symmetric layout and black hypertext for printing
%%               on paper
%%          two-side printing: twoside (default is one-sided printing)
%%               typeset in a wide margin on the binding side of the page and
%%               black hypertext. Use with print only.
%%

%% FOR USERS OF AMS PACKAGES:
%% * newtxmath used in this template loads amsmath, so
%%   you needn't load it. If you want to use options in amsmath, load it here, 
%%   before \setupthesisfonts below to pass the options to amsmath.
%% * If you want to use amsthm, load it here before \setupthesisfonts to avoid
%%   a clash with newtxmath.
%% * If using amsmath with options and you want to use amsthm, load amsthms
%%   after amsmath, as described in the amsthm documentation.
%% * Don't use amsbsym or amsfonts. The symbols [and macros] there are defined in
%%   newtxmath and so clash if used.
%\usepackage[options]{amsmath}
%\usepackage{amsthm}

%% DO NOT MOVE OR REMOVE \setupthesisfonts
\setupthesisfonts

%%
%% Add here the packges you need
%%
\usepackage{graphicx}


%% For tables that span multiple pages; used to split a paraphrasing example in
%% the appendix. If you don't need it, remove it.
\usepackage{longtable}

%% A package for generating Creative Commons copyright terms. If you don't use
%% the CC copyright terms, remove it, since otherwise undesired information may
%% be added to this document's metadata.
\usepackage[type={CC}, modifier={by-nc-sa}, version={4.0}]{doclicense}
%% Find below three examples for typesetting the CC license notice.

%%%%%%%%%%%%%%%%%%%%%%%%%% FOR THOSE WHO USE BIBLATEX %%%%%%%%%%%%%%%%%%%%%%%%%%
%% Package to use BibLaTeX with some settings. Adjust and add BibLaTeX settings
%% to suit your needs. See the package documentation for available options.
%% The bibliography printing and associated commands are below between the
%% conclusions and the appendix.
%%
\usepackage[
backend=biber,
style=numeric-comp, % citations and references are numerical (Vancouver, IEEE)
sorting=nyt, % references are in alphabetical order sorted in the order name,
             % year, and title.
% sorting=none, % references listed in the order of citation
firstinits=true, % show initial of first name in bibliography
urldate=long % date is expressed as Month dd, yyyy
]{biblatex}
\addbibresource{../references/refs.bib}
%%
%%%%%%%%%%%%%%%%%%%%%%%%%%%%%% END BIBLATEX STUFF %%%%%%%%%%%%%%%%%%%%%%%%%%%%%%

%% Edit to conform to your degree programme
%% Capitalise the words in the name of the degree programme: it's a name
\degreeprogram{Computer, Communication and Information Sciences}
%%

%% Your major
%%
\major{Computer Science}
%%

%% Choose one of the three below
%%
%\univdegree{BSc}
\univdegree{MSc}
%\univdegree{Lic}
%%

%% Your name (self explanatory...)
%%
\thesisauthor{Aleksi Elias Kääriäinen}
%%

%% Your thesis title and possible subtitle comes here and possibly, again,
%% together with the Finnish or Swedish abstract. Do not hyphenate the title
%% (and subtitle), and avoid writing too long a title. Should LaTeX typeset a
%% long title (and/or subtitle) unsatisfactorily, you might have to force a
%% linebreak using the \\ control characters. In this case...
%% * Remember, the title should not be hyphenated!
%% * A possible 'and' in the title should not be the last word in the line; it
%%   begins the next line.
%% * Specify the title (and/or subtitle) again without the linebreak characters
%%   in the optional argument in box brackets. This is done because the title
%%   is part of the metadata in the pdf/a file, and the metadata cannot contain
%%   linebreaks.
%%
\thesistitle{Hallucination Detection and Prevention in CS Education in AI-generated content}
%\thesistitle[Title of the thesis]{Title of\\ the thesis}
%%
%% Either remove or leave \thesissubtitle{} empty if you don't use it
%%
%\thesissubtitle{A possible subtitle}%
%\thesissubtitle[Subtitle of the thesis]{Subtitle of\\ the thesis}
%\thesissubtitle{}

%%
\place{Espoo}
%%

%% The date for the bachelor's thesis is the day it is presented
%%
\date{\today}
%%

%% Thesis supervisor
%% Note the "\" character in the title after the period and before the space
%% and the following character string.
%% This is because the period is not the end of a sentence after which a
%% slightly longer space follows, but what is desired is a regular interword
%% space.
%%
\supervisor{Prof.\ Juho Leinonen}
%%

%% Advisor(s)---two at the most---of the thesis. Check with your supervisor how
%% many official advisors you can have.
%%
\advisor{Ms Evanfiya Logacheva (MSc)}
%%

%% If you do your thesis work in a company of other institute, give the name of
%% the company or instution here. Otherwise, leave the macro empty, comment it
%% out, or remove it. This will remove this field from the abstract page.
%%
%\collaborativepartner{Company or institute name (if relevant)}%
%%

%% Aaltologo: syntax:
%% \uselogo{?|!|'|aalto?|aalto!|aalto'|<empty>}
%% The logo language is set to be the same as the thesis language.
%%
%\uselogo{?}
%\uselogo{!}
\uselogo{'}
%\uselogo{aalto?}
%\uselogo{aalto!}
%\uselogo{aalto'}
%\uselogo{}
%%

%%%%%%%%%%%%%%%%%%               COPYRIGHT TEXT               %%%%%%%%%%%%%%%%%%
%%%%%%%%%%%%%%%%%%%%%%%%%%%%%%%%%%%%%%%%%%%%%%%%%%%%%%%%%%%%%%%%%%%%%%%%%%%%%%%%

%% Copyright of a work is with the creator/author of the work regardless of
%% whether the copyright mark is explicitly in the work or not. You may, if you
%% wish---we encourage you to do so---publish your work under a Creative
%% Commons license (see creativecommons.org), in which case the license text
%% must be visible in the work. Write here the copyright text you want using the
%% macro \copyrighttext, which writes the text into the metadata of the pdf file
%% as well.
%%
%% Syntax:
%% \copyrigthtext{metadata text}{text visible on the page}
%%
%% CHOOSE ONE OF THE COPYRIGHT NOTICE STYLES BELOW.
%% IF USING THE CC TERMS, CHOOSE THE LICENSE YOU WANT TO USE.
%% The different CC licenses are listed at 
%% https://creativecommons.org/about/cclicenses/.
%% If you use the icons from the doclicense.sty package, add the package above
%% (\usepackage{doclicense}).
%% IMPORTANT NOTE!! Manually write the CC text in the \copyrighttext metadata
%% text field.
%%
%% NOTE: In the macros below, the text written in the metadata must have a
%% \noexpand macro before the \copyright special character. When not in pdf/a
%% mode (i.e. a-1b or a-2b are not specified in \documentclass), two \noexpands
%% are required in the metadata text to correctly render the copyright mark in
%% the pdf metadata. In pdf/a mode one \noexpand suffices.
%%
%% EXAMPLE OF PLAIN COPYRIGHT TEXT
%% The macros \copyright and \year below must be separated by the \ character 
%% (space chacter) from the text that follows. The macros in the argument of the
%% \copyrighttext macro automatically insert the year and the author's name.
%% (Note! \ThesisAuthor is an internal macro of the aaltothesis.cls class file).
%%
%\copyrighttext{Copyright \noexpand\textcopyright\ \number\year\ \ThesisAuthor}
%{Copyright \textcopyright{} \number\year{} \ThesisAuthor}
%%
%% Of course, the same text could have simply been written as
%% \copyrighttext{Copyright \noexpand\copyright\ 2018 Eddie Engineer}
%% {Copyright \copyright{} 2022 Eddie Engineer}
%%
%% EXAMPLES OF CC LICENSE: different ways to display the same license
%% 1. A simple Creative Commons license text with a link to the copyright notice:
%\copyrighttext{\noexpand\textcopyright\ \number\year. This work is 
%	licensed under a CC BY-NC-SA 4.0 license.}{\textcopyright{} 
%	\number\year. This work is licensed under a 
%	\href{https://creativecommons.org/licenses/by-nc-nd/4.0/}{CC BY-NC-SA 4.0} 
%	license.}
%
%% To get the URL of the license of your choice, go to 
%% https://creativecommons.org/about/cclicenses/, click on the chosen license
%% you want to use, and copy-and-paste the URL in the macro \href above.
%%
%% 2. A short Creative Commons license text containing the respective CC icons
%% (requires the package doclicense.sty to be added in the preamble as done
%% above) and a link to the corresponding Creative Commons license webpage (see
%% the doclicense package documentation for other license icons):
%\copyrighttext{\noexpand\textcopyright\ \number\year. This work is licensed
%	under a CC BY-NC-SA 4.0 license.}{
%	\parbox{95mm}{\noindent\textcopyright\ \number\year. \doclicenseText} 
%	\hspace{1em}\parbox{35mm}{\doclicenseImage}
%}
%%
%% 3. An expanded Creative Commons license text containing the respective CC
%% icons text and as generated by the doclicense.sty package (the license is set
%% via package options in \usepackage[options]{doclicense} above; see the
%% doclicense package documentation for other license texts and icons):
\copyrighttext{\noexpand\textcopyright\ \number\year. This work is 
	licensed under a Creative Commons "Attribution-NonCommercial-ShareAlike 4.0 
	International" (BY-NC-SA 4.0) license.}{\noindent\textcopyright\ \number
	\year \ \doclicenseThis}
%%%%%%%%%%%%%%%%%%%%%%%%%%%%%%%%%%%%%%%%%%%%%%%%%%%%%%%%%%%%%%%%%%%%%%%%%%%%%%%%


%% The English abstract:
%% All the details (name, title, etc.) on the abstract page appear as specified
%% above.
%% Thesis keywords:
%% Note! The keywords are separated using the \spc macro
%%
\keywords{Liiba\spc laaba\spc foo\spc 
bar}
%%

%% The abstract text. This text in one paragraph is included in the metadata of
%% the pdf file as well as the abstract page. To have paragraphs in your
%% abstract rewrite it in the abstarct environment as described below.
%%
\thesisabstract{%
The abstract is a short description of the essential contents of the thesis
usually in one paragraph: what was studied and how and what were the main
findings. For a Finnish thesis, the abstract should be written in both Finnish
and English; for a Swedish thesis, in Swedish and English. The abstracts for
English theses written by Finnish or Swedish speakers should be written in
English and either in Finnish or in Swedish, depending on the student’s language
of basic education. Students educated in languages other than Finnish or Swedish
write the abstract only in English. Students may include a second or third
abstract in their native language, if they wish. 
The abstract text of this thesis is written on the readable abstract page as
well as into the pdf file's metadata via the thesisabstract macro (see the 
comment in the TeX file). Write here the text that goes into the metadata. The 
metadata cannot contain special characters, linebreak or paragraph break 
characters, so these must not be used here. If your abstract does not contain 
special characters and it does not require paragraphs, you may take advantage of
the abstracttext macro (see the comment in the TeX file below). Otherwise, the 
metadata abstract text must be identical to the text on the abstract page.
}

%% You can prevent LaTeX from writing into the xmpdata file (it contains all the 
%% metadata to be written into the pdf file) by setting the writexmpdata switch
%% to 'false'. This allows you to write the metadata in the correct format
%% directly into the file thesistemplate.xmpdata.
%\setboolean{writexmpdatafile}{false}


%% All that is printed on paper starts here
%%
\begin{document}

%% Create the coverpage
%%
\makecoverpage

%% Typeset the copyright text.
%% If you wish, you may leave out the copyright text from the human-readable
%% page of the pdf file. This may seem like a attractive idea for the printed
%% document especially if "Copyright (c) yyyy Eddie Engineer" is the only text
%% on the page. However, the recommendation is to print this copyright text.
%%
\makecopyrightpage

\clearpage
%% Note that when writing your thesis in English, place the English abstract
%% first followed by the possible Finnish or Swedish abstract.

%% Abstract text
%% All the details (name, title, etc.) on the abstract page appear as specified
%% above. Add your abstarct text with paragraphs here to have paragraphs in the
%% visible abstract page. Nonetheless, write the abstarct text without
%% paragraphs in the macro \thesismacro so that it is added to the metadata as
%% well.
%%
\begin{abstractpage}[english]
    \abstracttext{}
\end{abstractpage}

%% The text in the \thesisabstract macro is stored in the macro \abstractext, so
%% you can use the text metadata abstract directly as follows:
%%
%\begin{abstractpage}[english]
%	\abstracttext{}
%\end{abstractpage}

%% Force a new page so that the possible Finnish or Swedish abstract does not
%% begin on the same page
%%

\dothesispagenumbering{}

%% Preface
%%
%% This section is optional. Remove it if you do not want a preface.
\mysection{Preface}
Liirumlaarum

\vspace*{\fill}
Otaniemi, \today \\

\vspace{5mm}
{\hfill Aleksi E.\ Kääriäinen \hspace{1cm}}

%% Force a new page after the preface
%%
\newpage


%% Table of contents. 
%%
\thesistableofcontents


%% Symbols and abbreviations
\mysection{Symbols and abbreviations}

\subsection*{Symbols}

\begin{tabular}{ll}
place & holder \\
\end{tabular}

\subsection*{Operators}

\begin{tabular}{ll}
place & holder \\
\end{tabular}

\subsection*{Abbreviations}

\begin{tabular}{ll}
CS      & Computer Science \\
LM      & Language Model \\
NLP     & Natural Language Processing \\
LLM     & Large Language Model \\
ICL     & In-Context Learning \\
PLM     & Pre-trained Language Model \\
SFT     & Supervised Fine-Tuning \\
RLHF    & Reinforcement Learning from Human Feedback \\
%ML      & Machine Learning \\
%AI      & Artificial Intelligence \\
\end{tabular}


%% \clearpage is similar to \newpage, but it also flushes the floats (figures
%% and tables).
%%
\cleardoublepage

%% Text body begins. 
%%
\section{Introduction}
\label{sec:intro}

%% Leave page number of the first page empty
%% 
\thispagestyle{empty}

\textbf{TODO: citations in intro! along with text improvements }

The subsequent advances in Language Models (LMs) have significantly extended the capabilities of computational Natural Language Processing (NLP) and text generation. Modern LMs are commonly transformer-based and pretrained on a Web-scale text corpora. These models are usually called Large Language Models (LLMs). LLMs, such as GPT-4 and Llama 4, have been shown to excel at a number of text-related tasks, including the generation of program code \cite{Minaee2024-fr}.

Computer Science (CS) education and programming skills are usually obtained by completing sets of various exercises. These exercises are typically created by hand by the teaching staff. Distributing the same set of exercises to a large group of students may lead to the students sharing the answers, which hinders learning, or the lack of relevant context in the exercises might not inspire students to complete them. Personalizing and contextualizing the exercises has been shown to increase student's engagement and improve the learning outcome. However, personalizing each exercise is a laborious task and is unfeasible for most CS teachers.

Since the emergence of LLMs, researchers, particularly in the field of CS education, have been interested in the models' ability to generate contextualized text, including computer program code, that is nearly indistinguishable from content written by humans. The use of LLMs in the creation of educational materials have a number of advantages compared to creating each exercise by hand: LLMs decrease the amount of time needed to create each exercise, they allow deep personalization of the exercises and can also be used to create the supporting artifacts in the assignments, including starter code, hints, and example solutions. Time-effectiveness and personalization are not the only important factors in creating educational content, however, it is also crucial to ensure that the materials offered to students are of high quality.

Various aspects of the training data, the training phase, and inference make LLMs prone to hallucinations. A hallucination in the context of LLMs is a model-generated output that is plausible but factually false \cite{10.1145/3703155}. Hallucinations decrease the quality of the generated content, and in an educational environment, may make the content unusable. Manually checking each output for erroneous content is time-consuming and prone to human-error. For this reason, it would be highly advantageous to mitigate hallucinations produced or to have the ability to detect and correct output containing hallucinations. 

One promising method of preventing hallucinations is in-context learning (ICL). ICL is the practice of introducing incorrect examples for the model as a demonstration and letting the model learn from them \cite{dong-etal-2024-survey}. ICL has shown great promise in reasoning-heavy tasks, such as mathematical reasoning \cite{an2024learningmistakesmakesllm}. Extending ICL to the generation of program code and programming exercises could decrease the amount of human input needed to create programming exercises.

\textbf{TODO: define scope!} This thesis investigates possible methods for detecting, preventing, or reducing hallucinations in educational CS content generated using LLMs. The study provides an empirical comparison of performance between implicit and explicit in-context learning in LLMs used in the generation of CS education content. The main research questions in this thesis are as follows:

\begin{itemize}
    \item \textbf{RQ1: How effective are state-of-the-art LLMs in generating error-free programming exercises?} This includes researching the effect of the prompt used in the generation, optimizing the prompt structures, and analyzing the faithfulness and factuality of the output.
    \item \textbf{RQ2: Can LLMs learn from previously produced erroneous content, producing higher quality content?} The objective of this research question is to find out whether in-context learning can be extended to the generation of CS education material and how much does in-context learning improve the output quality.
    \item \textbf{RQ3: Which method of ICL outperforms the other in generating CS education materials, implicit or explicit?} If the answer to the second research question is yes, the next point of interest is to compare the performance of the two different types of ICL.
\end{itemize}

These questions determine the course and content of the study. The thesis is structured as follows:

\begin{itemize}
    \item \textbf{Background:} In this section, existing relevant literature is reviewed and past studies are explained in order to understand the current state of research and bring the readers to a common level of knowledge on the subject. The first topic covered is the Fundamentals of Large Language Models, then Hallucinations in LLMs. After that, a brief explanation on Hallucination Mitigation Methods, and LLMs in CS Education, and finally, existing research in Automatic exercise generation is covered.
    \item \textbf{Research material and methods:} \textbf{TODO: content}
    \item \textbf{Results:} \textbf{TODO: content}
    \item \textbf{Conclusions:} \textbf{TODO: content}
\end{itemize}

%% In a thesis, every section/chapter starts a new page, hence the \clearpage
\clearpage

\section{Background}

This chapter explains the \textbf{TODO: finish paragraph!}

\subsection{Fundamentals of Large Language Models}

Large Language Models belong to the family of Pre-trained Language Models (PLMs). PLMs are task-agnostic, neural-network based, stochastic text prediction functions, trained on very large text corpora. The models operate by outputting the most likely next unit of text given a snippet of text as input. In essence, LLMs are large-scale PLMs, having tens to hundreds of billions of parameters. However, the distinction between the two is necessary, since LLMs have far stronger language understanding and generation abilities than PLMs, and also exhibit capabilities that are not present in PLMs, such as ICL, instruction following, and multi-step reasoning \cite{Minaee2024-fr}.

\begin{figure}[h]
\centering
\includegraphics[width=0.5\textwidth]{latex/images/transformer_architecture.png}
\caption{The Transformer architecture \cite{DBLP:journals/corr/VaswaniSPUJGKP17}}
\label{fig:transformer}
\end{figure}

Most LLMs are based on the Transformer architecture, introduced in \cite{DBLP:journals/corr/VaswaniSPUJGKP17}. The most important innovation in the Transformer model is the application of the attention mechanism, which captures long-term contextual dependencies in the input text, to the parallelizable architecture, depicted in figure \ref{fig:transformer}. This architecture allows for efficiently pre-training very large LMs that have been shown to outperform other existing Neural Language Models, such as Recurrent and Convolutional Language Models \cite{DBLP:journals/corr/VaswaniSPUJGKP17}.

The Transformer model consists of an encoder stack and a decoder stack, pictured on the left and right sides of figure \ref{fig:transformer}, respectively. The encoder maps the input from a sequence of symbols $(x_1, \cdots, x_n)$, to a continuous representation of the symbols $z = (z_1, \cdots, z_n)$. Given $z$, the decoder generates an output sequence $y = (y_1, \cdots, y_m)$ of symbols one symbol at a time. The decoder is auto-regressive, which means that it consumes previously generated symbols $(y_1, \cdots, y_{k-1})$ as an additional input when generating the next $y_k$ \cite{DBLP:journals/corr/VaswaniSPUJGKP17}. Such \textit{Encoder-Decoder models} are best suited for sequence-to-sequence language tasks that are conditioned on input, such as text summarization and language translation tasks \cite{Minaee2024-fr}.

However, many LLMs, most notably the GPT-family and the Llama-family, omit the encoder completely and are strictly \textit{Decoder-Only models} \cite{Minaee2024-fr}. These models function purely as auto-regressive language models, predicting the next token based only on previous tokens. Decoder-Only models have been shown to excel in different text generation tasks, such as answering questions and generating program code \cite{Chen2021EvaluatingLL}.

\subsubsection{Training Stages of LLMs}

In order for any neural LM to be able to take text as input, the text first has to be converted to a form that the LM can understand. Tokenization is the practice of breaking down text into smaller chunks, called tokens. A token is the basic unit of operation in LMs. The most basic tokenizers split text on whitespace, but more advanced tokenizers break each individual word into subwords, or even into single characters. Using smaller tokens enables combining multiple tokens to form words, effectively increasing the size of the vocabulary and decreasing the chance of out-of-vocabulary encounters \cite{Minaee2024-fr}.

Most LLMs, and the Transformer, convert tokens further into an embedding, which is a high-dimensional vector representation of the token. This is usually done in the learnable embedding layer \cite{DBLP:journals/corr/VaswaniSPUJGKP17}, which means that the model optimizes the vector representation of each token, allowing capture and measurement of word similarities and distances. Additionally, positional encoding is included in the embeddings, which allows the model to learn positional dependencies and to carry contextual information spread out in the input sequence.

LLMs acquire their capabilities through a three phase training process:

\begin{enumerate}
    \item \textbf{Pre-training:} During this phase, the LLM undergoes autoregressive prediction of subsequent tokens in text sequences. The self-supervised training on large textual corpora, often covering a significant portion of textual content on the internet, results in the LLM acquiring knowledge of language syntax, world knowledge, and reasoning abilities \cite{10.1145/3703155}. However, during this phase, the model also learns biases, infactualities, and half-truths present in the training data \cite{Kalai2025-fl}.
    
    \item \textbf{Supervised Fine-Tuning:} While pre-training focuses on refining the text completion and language understanding abilities of the model, Supervised Fine-Tuning (SFT) focuses on enchancing the model's task completion ability. During SFT, the model is fed pairs of $(\texttt{instruction, output})$, where \texttt{instruction} (e.g., write an email to person $X$ on the topic $Y$) is the input, instructing the model to complete a specific task, and \texttt{output} is the desired outcome of the task. The objective of using $(\texttt{instruction, output})$ pairs is to constrain the model's outputs to a desired domain of knowledge and align the model with specific response characteristics \cite{zhang2025instructiontuninglargelanguage}. The datasets used in this phase are usually constructed from existing annotated natural language datasets. Examples of such datasets are Flan \cite{longpre2023flancollectiondesigningdata} and P3 (Public Pool of Prompts) \cite{sanh2022multitaskpromptedtrainingenables}. Another possible way of creating a dataset of $(\texttt{instruction, output})$ pairs is to employ an existing LLM to generate such pairs. This approach has been explored for instance by \textcite{wang2023labelwordsanchorsinformation}.

    \begin{figure}[h]
    \centering
    \includegraphics[width=1\textwidth]{latex/images/rlhf.png}
    \caption{The 3-step pipeline of RLHF used by \textcite{ouyang2022traininglanguagemodelsfollow} to fine-tune an LLM. Note that Step 1 is analogous to SFT.}
    \label{fig:rlhf}
    \end{figure}

    \item \textbf{Reinforcement Learning from Human Feedback:} Reinforcement Learning from Human Feedback (RLHF), sometimes called Alignment and sometimes included in the SFT phase, is the final stage of training an LLM. RLHF is the process of aligning the output of the model to the preference of human users. Figure \ref{fig:rlhf} shows the progress of RLHF: the model is first fine-tuned in the SFT phase. Then, a reward model is trained on human-ranked responses and finally, using a policy optimization algorithm, such as PPO \cite{zhang2025instructiontuninglargelanguage}, the reward is maximized. This type of training allows the model to learn high-level objectives, such as values, tone, and behavior patterns. RLHF is also used to mitigate the possible harm the model could do once deployed in the real world \cite{ouyang2022traininglanguagemodelsfollow}. These unwanted responses include biased outputs, leaking private information, presenting misinformation as factual content, and generating harmful and toxic content \cite{weidinger2021ethicalsocialrisksharm}.
\end{enumerate}

\subsubsection{Decoding strategies?}
\subsubsection{Cost-Effective Training/Inference, Adaptation \& Compression?}

\subsection{Hallucinations in LLMs}

Hallucinations in the context of LLMs are $\cdots$

\subsubsection{Hallucination types}

\begin{figure}[h]
    \centering
    \includegraphics[width=1\textwidth]{latex/images/hallucination_types.png}
    \caption{Examples of each type of LLM hallucination presented by \textcite{10.1145/3571730}.}
    \label{fig:hallucination_types}
\end{figure}

\subsubsection{Causes of hallucinations}

\subsection{Hallucination Mitigation Methods}
\subsection{LLMs in CS Education}
\subsection{Automatic exercise generation?}

\clearpage

\section{Research material and methods}

This part is the core of your work, where you explain the methodological choices
you made, its limitations, how you pick your research material or subjects, the 
implementation of your study and the methods used. This section determines the 
methodological strengths and weaknesses of your thesis. Any earlier description 
of the method should limit itself to work done earlier by others. Here you tell 
your reader what you have done.

\begin{itemize}
        \item Feedback loop?
        \begin{itemize}
            \item Feed demonstration examples to model
            \item Prompt
            \item Output
            \item Decision: Factual output?
                \begin{itemize}
                    \item Yes -> Continue
                    \item No -> Add set [prompt, output, label, (reasoning)] to demonstration examples
                \end{itemize}
        \end{itemize}
\end{itemize}

\clearpage

\section{Results}

Present the results of your study here and answer the research questions, asked 
earlier in the thesis (in the introduction, perhaps), this study strives to 
answer. The scientific value of your work is measured by the results you obtain 
along with the arguments you give to back the answers to your research 
questions.

Be critical of the significance of your results. You may critically scrutinise 
the results and your interpretation of the results here, or you may do so later 
in the chapter with the discussion of your work or in the conclusions part.

This part should discuss how reliable the data used in the study are. You may 
discuss the reliability of the conclusions drawn from the study either in this 
chapter or later in the discussions part. You may have the discussion in a 
chapter of its own, separate from the summary or conclusions.


\clearpage

\section{Summary/Conclusions}
\label{sec:summary}

This is where you tie up any loose ends. Tell your reader briefly and clearly 
what you have done, what you have discovered, and the value of your discovery 
in the context of similar work done earlier. Draw clear conclusions regarding 
the research problem, sub-problems or hypotheses. You also discuss future lines 
of study and new questions your study might have posed.

As the author of the thesis, you alone are responsible for ensuring that the 
layout, form and structure of your thesis adheres to the guidelines outlined by 
your school. This template aims to help you meet these requirements.



\clearpage
%% Bibliography / list of references
%%

%%%%%% FOR THOSE WHO USE BIBLATEX %%%%%%
% Redefine 'visited on' to 'Accessed on' the 
%\DeclareFieldFormat{urldate}{%
%	(Accessed on %
%	\mkbibmonth{\thefield{urlmonth}}\addspace%
%	\thefield{urlday}\addcomma \addspace      %
%	\thefield{urlyear}\isdot)}

%\nocite{*} % print uncited references in the bibliography
\printbibliography[heading=bibintoc] %, add the title to the table of 
%                                                 contents title={References}.

%%%%%%%%%% END BIBLATEX STUFF %%%%%%%%%%

%% The hand-written bibliography
%\thesisbibliography % Required to get the bibliography title in toc and to get
                     % the page number hyperlink to the page correct.

%\begin{thebibliography}{99}
%
%  \bibitem{aaltolib} Citation Guide: Making a bibliography, \textit{Aalto 
%  	University Learning Centre}. Online article. Available  
%    \url{https://libguides.aalto.fi/c.php?g=410674&p=2797572}
%    (accessed on 14.7.2021)
%
%  \bibitem{Bringhurst} Bringhurst, R., \textit{Horizontal Motion. The Elements 
%  	of Typographic Style}, Point Roberts, WA: Hartley \& Marks, 1992. p. 26, 
%    pp.\ 25--36. Also available online as version 3.0 at  
%    %\url{https://smallpressblog.files.wordpress.com/2017/11/bringhurstelementsselections1.pdf} (accessed on 7 May 2021).
%
%  \bibitem{Dyson} Dyson, M. C., and Kipping, G. J., ``The Effects of Line Length 
%    and Method of Movement on Patterns of Reading from Screen,'' 
%    \textit{Visible Language,} vol.~2, no.~2, pp. 150--181, 1998.
%
%  \bibitem{Wikilinelength} Wikipedia contributors, ``Line length,'' 
%    \textit{Wikipedia: The Free Encyclopedia}, Wikimedia Foundation, Inc., 
%    22 July 2004.
%    \url{https://en.wikipedia.org/w/index.php?title=Line_length&oldid=997524503}
%    (accessed on 7 May 2021).
%
%\end{thebibliography}

%% Appendices
%% If you don't have appendices, your thesis ends here. Remove \clearpage,
%% \thesisappendix and the following text below. The last command of this file
%% is \end{document}.
\clearpage

\thesisappendix

\section{Contents of an appendix}
\label{app:contents}

Appendices are not essential in a thesis, and so you must plan the content of 
your thesis as if it does not contain an appendix. The appendix cannot be used 
as a dumping ground for text and ideas from an overgrown thesis.

An appendix is an independent entity, even though it complements the thesis. 
So, the appendix is not, say, just a list or image or table, but contains 
explanatory text as well that indicates the purpose of its content. It can 
contain code listings, like the one below for a simplified list of commands to 
create an appendix.

The appendix can contain figures that do not fit in to complement the text in 
the thesis. The numbering of figures is like that of equations: see figure~\ref{appfig:refraction}.

The numbering of tables is like that for equations and figures, as is evident 
from the caption of table~\ref{apptab:schedule}.

%% Example of a table in the appendix. Note how h places the table in the
%% current position.
\begin{table}[htb]
	\centering
	\caption{Caption for the table.}
	\label{apptab:schedule}
	\sffamily% change the font in the table to sans serif
	\fbox{
		\begin{tabular}{lp{0.5\linewidth}}
			9.00--9.55  & Safety instructions on the use of laboratories\\
			9.55--10.00 & Transfer to the laboratory
		\end{tabular}}
\end{table}

\end{document}
